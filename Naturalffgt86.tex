% !TEX TS-program = pdflatex
% !TEX encoding = UTF-8 Unicode

% This is a simple template for a LaTeX document using the "article" class.
% See "book", "report", "letter" for other types of document.

\documentclass[11pt]{article} % use larger type; default would be 10pt

\usepackage[utf8]{inputenc} % set input encoding (not needed with XeLaTeX)

%%% Examples of Article customizations
% These packages are optional, depending whether you want the features they provide.
% See the LaTeX Companion or other references for full information.

%%% PAGE DIMENSIONS
\usepackage{geometry} % to change the page dimensions
\geometry{a4paper} % or letterpaper (US) or a5paper or....
% \geometry{margin=2in} % for example, change the margins to 2 inches all round
% \geometry{landscape} % set up the page for landscape
%   read geometry.pdf for detailed page layout information

\usepackage{graphicx} % support the \includegraphics command and options

% \usepackage[parfill]{parskip} % Activate to begin paragraphs with an empty line rather than an indent

%%% PACKAGES
\usepackage{booktabs} % for much better looking tables
\usepackage{array} % for better arrays (eg matrices) in maths
\usepackage{paralist} % very flexible & customisable lists (eg. enumerate/itemize, etc.)
\usepackage{verbatim} % adds environment for commenting out blocks of text & for better verbatim
\usepackage{subfig} % make it possible to include more than one captioned figure/table in a single float
% These packages are all incorporated in the memoir class to one degree or another...

\usepackage{amsmath}
\usepackage{algorithm}
\usepackage{algpseudocode}

%%% HEADERS & FOOTERS
\usepackage{fancyhdr} % This should be set AFTER setting up the page geometry
\pagestyle{fancy} % options: empty , plain , fancy
\renewcommand{\headrulewidth}{0pt} % customise the layout...
\lhead{}\chead{}\rhead{}
\lfoot{}\cfoot{\thepage}\rfoot{}

%%% SECTION TITLE APPEARANCE
\usepackage{sectsty}
\allsectionsfont{\sffamily\mdseries\upshape} % (See the fntguide.pdf for font help)
% (This matches ConTeXt defaults)

%%% ToC (table of contents) APPEARANCE
\usepackage[nottoc,notlof,notlot]{tocbibind} % Put the bibliography in the ToC
\usepackage[titles,subfigure]{tocloft} % Alter the style of the Table of Contents
\renewcommand{\cftsecfont}{\rmfamily\mdseries\upshape}
\renewcommand{\cftsecpagefont}{\rmfamily\mdseries\upshape} % No bold!

%%% END Article customizations

\title{Natural Computing}
\author{ffgt86}
%\date{} % Activate to display a given date or no date (if empty),
         % otherwise the current date is printed 

\makeatletter
\def\BState{\State\hskip-\ALG@thistlm}
\makeatother

\begin{document}
\maketitle

\section{Relevant background material}

The Artificial Bee Colony (ABC) algorithm is an optimisation algorithm based on the foraging behaviour of the honeybee swarm. 

\clearpage

\section{Pseudocode}

\begin{algorithm}
\caption{My algorithm}\label{euclid}
\begin{algorithmic}[1]
\Procedure{ABC}{}

% The number of food sources
\State $S_n$: number of food sources
% The number of trials before abandoning a food sources
\State $limit$: number of trials before abandoning a food source
% The number of generations
\State $generations$: number of generations
% The number of cycles
\State $C_{max}$: number of cycles

\BState \emph{begin}:

\State \textit{//Initialisation}
% The current number of generations
\State $num\_eval \gets 0$

% This paper stresses the importance of measuring the number of fitness evaluations, NOT the number of generations
% So I'll have to modify it back to the original. But that's fine: it's an excellents starting point. 

% Generate a randomly distributed initial population of S_n solutions.
\For{$s \gets 1$ to $S_N$}         
	
	% Set the solution of employee s to a random solution           
	\State $X(s) \gets \textit{random solution}$
	% Compute the fitness of that random solution
	\State $f_s \gets f(X(s))$
	% Set the value of trial[s] to 0
	\State $trial(s) \gets 0$
	% Increment the number of evaluations that have been performed
	\State $num\_eval++$

\EndFor

\State \textit{//Employed bees phase}

\While{$c < C_{max}$}

	\For{$s \gets 1$ to $S_N$}         
		
		% Get a neighbouring solution  
		\State $x' \gets \textit{a new solution}$

		% Increment the number of evaluations
		%\State $num\_eval \gets {num\_eval + 1}

	\EndFor

\EndWhile


\EndProcedure
\end{algorithmic}
\end{algorithm}

\section{Natural language description}

\section{Details of experiments}

The paper used 5 classic optimisation benchmark functions, sourced from \cite{srinivasan_2003} to test the performance of ABC against PSO, PS-EA, and GA:

\begin{itemize}
	\item{Griewank: the Griewank function has a product term that introduce interdependence between variables. Algorithms that seek to optimise each variable independently will therefore do poorly.}
	\item{Rastrigin: the Rastrigin function produces many local, regularly distributed minima, so that an optimisation algorithm can be easily trapped.}
	\item{Rosenbrock: the Rosenbrock function produces a global minimum inside a long, narrow, parabolic valley. Converging is difficult. It is used to test the performance of an algorithm.}
	\item{Ackley: the Ackley function tests whether an algorithm effectively combines exploration and exploitation}
	\item{Schwefel: the Schwefel function produce a complex map, with a second-best minimum far from the global minimum, which itself is close to the bounds of the domain.}
\end{itemize}

The paper used the maximum number of generations as $500$, $750$, and $1000$ for dimensions of $10$, $20$, and $30$, respectively. A population size of 125 was used as per \cite{srinivasan_2003}

The GA used used single-point uniform crossover with probability $P_c = 0.95$, with individuals selected randomly. 
Single-point uniform crossover switches two genes at the same place in each individual. 


Gaussian mutation with probability $P_m = 0.1$, and a linear ranking fitness function.

The maximum number of cycles (MCN) is equivalent to the maximum number of generations, and the colony size equals the population size. $50\%$ of the colony were onlookers; $50\%$ were employees, and 1 was a scout.

Each experiment was repeated 30 times, with different random seeds. The mean function values of the best solutions found by the algorithms for different dimensions were recorded.

The performance of PS-EA deteriorated in optimising.

The ABC algorithm can both escape local optima and converge to the global optimum. The exploration faciliated by scouts is good for global optimisation, and the exploitation facilitated by onlookers and employees is good for local optimisation.

PS-EA (Particle Swarm - Evolutionary Algorithm) is a PSO / EA hybrid. It uses several mechanisms to adaptively update individuals based on the convergence rate or status of the algorithm in a particular iteration.


The paper found the ABC algorithm to be successful in optimising both multivariable and multimodal functions.



\section{Overview of results}

\end{document}
